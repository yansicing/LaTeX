% !Mode::"TeX:UTF-8"
\documentclass[twocolumn,landscape,UTF8]{ctexart}
\usepackage{lastpage}
%\usepackage{times} %use the Times New Roman fonts
\usepackage{color}
%\usepackage{placeins}
\usepackage{ulem}
\usepackage{titlesec}
\usepackage{graphicx}
\usepackage{colortbl}
\usepackage{listings}
\usepackage{makecell}
\usepackage{indentfirst}
\usepackage{fancyhdr}
\usepackage{setspace}       % 行间距
\usepackage{bm}             %\boldsymbol 粗体
% 数学
\usepackage{amsmath,amsfonts,amsmath,amssymb,times}
\usepackage{txfonts}
\usepackage{enumerate}     % 编号
\usepackage{tikz,pgfplots} % 绘图
\usepackage{tkz-euclide,pgfplots}
\usetikzlibrary{automata,positioning}
%\usepackage[paperwidth=18.4cm,paperheight=26cm,top=1.5cm,bottom=2cm,right=2cm]{geometry} % 单页
\usepackage[paperwidth=36.8cm,paperheight=26cm,top=2.5cm,bottom=2cm,right=2cm]{geometry}
\lstset{language=C,keywordstyle=\color{red},showstringspaces=false,rulesepcolor=\color{green}}
\oddsidemargin=0.5cm       % 奇数页页边距
\evensidemargin=0.5cm      % 偶数页页边距
%\textwidth=14.5cm         % 文本的宽度 单页
\textwidth=30cm            % 文本的宽度 单页

\newsavebox{\zdx}          % 装订线

\newcommand{\putzdx}{\marginpar{
		\parbox{1cm}{\vspace{-1.6cm}
			\rotatebox[origin=c]{90}{
				\usebox{\zdx}
		}}
}}

\newcommand{\blank}{\uline{\textcolor{white}{a}\ \textcolor{white}{a}\ \textcolor{white}{a}\ \textcolor{white}{a}\ \textcolor{white}{a}\ \textcolor{white}{a}\ \textcolor{white}{a}\ \textcolor{white}{a}\ \textcolor{white}{a}\ \textcolor{white}{a}\ \textcolor{white}{a}}}

\newcommand{\me}{\mathrm{e}}  % 定义 对数常数e,虚数符号i,j以及微分算子d为直立体。
\newcommand{\mi}{\mathrm{i}}
\newcommand{\mj}{\mathrm{j}}
\newcommand{\dif}{\mathrm{d}}
\newcommand{\bs}{\boldsymbol} % 数学黑体
\newcommand{\ds}{\displaystyle}
%通常我们使用的分数线是系统自己定义的分数线,即分数线的长度的预设值是分子或分母所占的最大宽度,如何让分数线的长度变长成,我们%可以在分子分母添加间隔来实现。如中文分式的命令可以定义为:
%\newcommand{\chfrac[2]}{\cfrac{\;#1\;}{\;#2\;}}
%\frac{1}{2} \qquad \chfrac{1}{2}

%选择题
\newcommand{\fourch}[4]{\\\begin{tabular}{*{4}{@{}p{3.5cm}}}
		(A)~#1 & (B)~#2 & (C)~#3 & (D)~#4\end{tabular}} % 四行
\newcommand{\twoch}[4]{\\\begin{tabular}{*{2}{@{}p{7cm}}}
		(A)~#1 & (B)~#2\end{tabular}\\\begin{tabular}{*{2}{@{}p{7cm}}}(C)~#3 &		(D)~#4\end{tabular}}  %两行
\newcommand{\onech}[4]{\\(A)~#1 \\ (B)~#2 \\ (C)~#3 \\ (D)~#4}  % 一行

\renewcommand{\headrulewidth}{0pt}
\pagestyle{fancy}
\begin{document} % 在begin前面加了一个空格以免出现显示错误,编译时应该去掉
\fancyhf{}
\fancyfoot[CO,CE]{\vspace*{1mm}第\,\thepage\,页 , 共 ~\pageref{LastPage} 页}
\sbox{\zdx}
{\parbox{27cm}{\centering
	时~间\underline{\makebox[34mm][c]{}}~
	年~龄\underline{\makebox[34mm][c]{}}~\CJKfamily{song} 性~别\underline{\makebox[44mm][c]{}}~\CJKfamily{song}
	姓~名\underline{\makebox[34mm][c]{}}~\\
	\vspace{3mm}
请在所附答题纸上空出密封位置,并填写相关信息。\\
%答题时学号
\vspace{1mm}
\dotfill{} 密\dotfill{}封\dotfill{}线\dotfill{} \\
	}}
	\reversemarginpar
	
\begin{spacing}{1.25}
\begin{center}
\begin{LARGE}
	
电子科技大学~\underline{~2019~-- 2020}\,学年第\,\underline{~1~}\,学期\\
\underline{\textbf{恋爱真心话}}期末试卷\\

\end{LARGE}
(开卷笔试 \quad  \quad 2020年01月30日 \quad 出题人:杨庆)\\

%(开卷笔试 \quad 杨庆\&张继 \quad 2020年01月30日 \quad 出题人:杨庆)\\
%(开卷笔试\ \ 杨庆\&张继\,2020年01月30日\,出题人:杨庆)\\

\begin{flushleft}
\ \textbf{注意事项:}\\
\quad 1. 本测试包括选择题、填空题、判断题、分析题、问答题共5种题型,满分100分;\\
\quad 2. 本测试没有标准答案,符合自己的内心想法即可;\\
\quad 3. 本测试仅供增进双方了解及互相交流,回答存在失误或考虑不全面无太大影响;\\
\quad 4. 本测试为开放答题,无固定时间和其他限制,需要双方自己商量;\\
\quad 5. 本测试回答完毕后发给对方,双方根据回答酌情给分,如存在表述不清可向对方解释。

\end{flushleft}

%\begin{itemize}
%	\item 1.本测试共包括选择题、填空题、判断题、分析题、问答题5种题型。 
%	\item 2.本测试没有标准答案,符合自己的内心想法即可。   
%	\item 3.本测试仅供增进双方了解及互相交流,回答存在失误或考虑不全面无太大影响。
%	\item 4.本测试为开放答题,无固定时间和其他限制,需要双方自己商量。
%\end{itemize} 

%\vspace{0.1cm}

\begin{tabular}{|m{0.03\textwidth}|*{8}{m{0.035\textwidth}|}p{0.04\textwidth}|}
\hline
\centering  题~号 & \centering 一 & \centering 二 & \centering 三 & \centering 四& \centering 五 % & \centering 六 & \centering 七 & \centering 八 & \centering 九 & \centering 十
& \centering 总~分 & \makecell{阅卷\\教师} \rule{0pt}{3mm} \\
\hline
\centering 分~数 &  &  &  &  &  &  &  % &  &  &  &
\rule{0pt}{8mm} \\\hline
				% \centering 计 &  &  &  &  &  &  &  &  &  &  & \\
				% \centering 分 &  &  &  &  &  &  &  &  &  &  & \\
				% \centering 人 &  &  &  &  &  &  &  &  &  &  & \\  \hline
\end{tabular}
\end{center}
\end{spacing}
\vspace{-0.5cm}
\setlength{\marginparsep}{1.7cm}
\putzdx %%装订线--奇页数
\vspace{1cm}
	\begin{spacing}{1.3}
		
		\section*{\hspace{5cm} 一、选择题~(每题~1 分,共~7 分)}
		\vspace{-2cm}
		\begin{tabular}{|p{0.05\textwidth}|p{0.05\textwidth}|}
			\hline
			% after \\: \hline or \cline{col1-col2} \cline{col3-col4} ...
			\centering 阅卷人& \\
			\hline
			\centering 得~~分 &  \\
			\hline
		\end{tabular}
		
\begin{enumerate}\setcounter{enumi}{0}
	
\item 情侣过了热恋期之后如何保持彼此间的感情?~(~~~~)
\onech
{两个人保持生活、学习、工作、思想状态等各方面的同步和协调,不能差异太大}
{培养共同的兴趣爱好,经常合作完成一件事情}
{给双方独处的空间和时间}
{准备小惊喜,保持仪式感}

\item 你最在乎另一半的特质是什么?~(~~~~)
\fourch{外貌身材}{思想性格}{能力素质}{未来发展}
			
\item 发生矛盾之后,谁应该先道歉?~(~~~~)
\onech
{最好是男方先低头,毕竟女孩比较感性一些}
{女方应该先低头,以唤起男方的同情心}
{谁理亏谁先低头}
{都应该主动向对方表明自己其实也有做得不对的地方}
			
\item 男女交往最忌讳什么?~(~~~~)
\onech
{把爱情的存在当成理所当然的事}
{看到别人分手了对自己的爱情也产生深深的怀疑}
{通过贬低对方显示自己的优越感}
{遇到问题埋在心里,对方不问就不说}

\item 男人对女人的误解有哪些?~(~~~~)
\onech
{女人年龄大了没结婚不是生理有问题就是心理有问题}
{女人没有男人好色,更追求精神享受}
{女人都比较现实和挑剔}
{男人对女人太好,女人就会不珍惜}

\item 女人对男人的误解有哪些?~(~~~~)
\onech  %\twoch 
{总有一个完美的男人在等我,只是我没遇到}
{男人大部分是靠下半身思考的动物}
{男的在很多方面都应该比我强}
{男的不主动联系我,就是不喜欢我}

\item 恋爱中你觉得有意思或可以尝试的事情有哪些?~(~~~~)
\onech
{一起逛菜市场,感受生活中的柴米油盐,你砍价,我付钱}
{教对方一件你擅长的事情,让他觉得“ WOW ”的事情}
{互相穿对方的衣服,再用上换脸软件自拍,看看两人有没有违和感哈哈哈}
{每天坚持说一次我爱你或亲昵的话,爱总不说,时间长了就再也不好意思说出口了}

\end{enumerate}
%\newpage

\section*{\hspace{5cm} 二、填空题~(每题~1 分, 共~ 10 分)}
\vspace{-1cm}

\begin{tabular}{|p{0.05\textwidth}|p{0.05\textwidth}|}
	\hline
	% after \\: \hline or \cline{col1-col2} \cline{col3-col4} ...
	\centering 阅卷人& \\
	\hline
	\centering 得~~分 &  \\
	\hline
\end{tabular}
\begin{enumerate}\setcounter{enumi}{7}
	
\item 你们第一次见面的时间:~\underline{\hbox to 40mm{}}.
	
\item 你们第一次加微信时间:~\underline{\hbox to 40mm{}}.
	
\item 你们第一次看电影时间:~\underline{\hbox to 40mm{}}.
	
\item 对方的生日:~\underline{\hbox to 40mm{}}.
	
\item 你们第一次认识的地点:~\underline{\hbox to 40mm{}}.
	
\item 对方送你的第一件礼物是什么:~\underline{\hbox to 40mm{}}.
	
\item 你们逛宜家商场最喜欢抚摸哪件物品:~\underline{\hbox to 40mm{}}.
	
\item 去生命奥秘博物馆哪个肌肉组织说了很多次:~\underline{\hbox to 40mm{}}.
	
\item 对方曾在老年痴呆测试中的提到哪几个词语:~\underline{\hbox to 40mm{}}.
	
\item 对方曾经带你见过哪些人:~\underline{\hbox to 40mm{}}.
	

\end{enumerate}

{\color{red}开心一刻:有一个腼腆的男孩向一女孩问到:你喜欢什么样的男孩?女孩说:投缘的。男孩哭丧着脸问到:头扁的不行吗?} 
\newpage
\putzdx %%装订线--奇页数

\section*{\hspace{4.5cm} 三、判断题:~(每题~1 分, 共~10 分)}
\vspace{-1cm}
\begin{tabular}{|p{0.05\textwidth}|p{0.05\textwidth}|}
\hline
			% after \\: \hline or \cline{col1-col2} \cline{col3-col4} ...
\centering 阅卷人& \\
\hline
\centering 得~~分 &  \\
	\hline
\end{tabular}
\begin{enumerate}\setcounter{enumi}{17}
			
\item 美国心理学家斯腾伯格提出的爱情理论,认为爱情由“激情、亲密、承诺”三个基本成分组成。 \hfill$(~~~~~~~~)$

\item 好的感情需要男女双方共同培养和经营,且爱情是双需,不是单需。 \hfill $(~~~~~~~~)$

\item 如果对方是个思想成熟的人,你的懂事,体谅是加分项,如果对方是个滥情人,这些美好品质就是平庸无趣。 \hfill $(~~~~~~~~)$

\item 把对方对你的好忽略掉,你还愿意和他在一起,这才是真爱。 \hfill$(~~~~~~~~)$

\item 恋爱就是场考试,选人考验你的辨别能力,相处考验你的解决能力,放弃考验你的取舍能力。 \hfill$(~~~~~~~~)$

\item 感情是靠吸引的,不是靠感动的。 \hfill$(~~~~~~~~)$

\item 你的另一半应该是一个能真正让你进步的人,这种进步可以是思想上,能力上,情绪上等各种方面。 \hfill$(~~~~~~~~)$

\item 对方爱你的方式,是你爱他的方式中互相影响,潜移默化的结果。 \hfill$(~~~~~~~~)$

\item 主动会掉价,克制才能长久。 \hfill$(~~~~~~~~)$

\item 你认真对待爱情,爱情也会认真对待你。 \hfill $(~~~~~~~~)$
	
\end{enumerate}

\vspace{0.5cm}
		
\section*{\hspace{2cm} 四、分析题~(6'+7'+10')}
		\vspace{-2cm}
		\begin{tabular}{|p{0.05\textwidth}|p{0.05\textwidth}|}
			\hline
			% after \\: \hline or \cline{col1-col2} \cline{col3-col4} ...
			\centering  阅卷人&  \\
			\hline
			\centering 得~~分 &  \\
			\hline
		\end{tabular}
		\begin{enumerate}\setcounter{enumi}{27}
		\vspace{0.5cm}
		
\item 一段感情会经历几个阶段,简要分析各个阶段的特点,并给出两个人是否能一直保持在热恋期的理由。(本题满分6分)
		
%答:~(1) 热恋期,了解期,整合期 \dotfill{}(2')

\item 如果有一天突然对方心情不好了,或者你们发生矛盾了,你会怎么做?
(本题满分7分)
		
\item 有女(男)朋友,但又遇到更喜欢的,试分析你该怎么办,并说明理由。(本题满分10分)	

\end{enumerate}
\vspace{0.5cm}

%\newpage
\section*{\hspace{5cm}五、问答题~(每题~2 分, 共~ 50 分)}
\vspace{-1cm}
\begin{tabular}{|p{0.05\textwidth}|p{0.05\textwidth}|}
\hline
			% after \\: \hline or \cline{col1-col2} \cline{col3-col4} ...
\centering 阅卷人& \\
\hline
\centering 得~~分 &  \\
\hline
\end{tabular}
		
\begin{enumerate}\setcounter{enumi}{30}
	
\item 讲讲你和对方的恋爱过程的吧。
			
\item 说说你对自己的认识,如自己的成长环境,性格,爱好等。

\item 谈谈你对对方的认识。

\item 你对 TA 的第一印象是什么样的?

%\item 对方喜欢吃的饭菜是什么?

%\item 给对方一条人生的建议。

\item 对方的什么特点你比较喜欢?

\item 对方做过什么事情令你比较感动?

\item 遇到对方,你觉得发生了什么变化?

\item 你们一起去过哪些地方,做过什么事情?

\item 你觉得情侣还可以一起做的事情有哪些?

\item 在没谈恋爱之前,你的恋爱标准是什么?(比如:性格、思想、学历、身高、身材等)

\item 一般来说,女人和男人各自会面临什么压力。面对未来的压力,你有什么想法?

\item 感情中你最不能接受的是什么?

\item 常见分手的原因有哪些?

\item 对方的优缺点有哪些?

\item 如果能有一个机会满足你,你最想和另一半一起做什么?

\item 你有什么是你自己挺满意和喜欢的?

\item 你觉得自己父母的性格是什么样的?

\item 你父母对你未来另一半有什么想法或要求?

\item 谈谈自己的爱情观、人生观、价值观、世界观。

\item 你觉得该如何平衡好工作和家庭?

\item 你觉得恋爱有哪些话题可以讨论?

\item 你最喜欢哪一部电视剧,电影,歌曲,并简要说明原因。

\item 有哪些你认为比较重要的思想或名言?

\item 男:“我喜欢一个人。”请问他可能表达的是什么意思?

\item 给对方写一句肉麻的话。


\end{enumerate}
	
\end{spacing}
\clearpage
\end{document}
